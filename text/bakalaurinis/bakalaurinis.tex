\documentclass{VUMIFPSbakalaurinis}
\usepackage{algorithmicx}
\usepackage{algorithm}
\usepackage{algpseudocode}
\usepackage{amsfonts}
\usepackage{amsmath}
\usepackage{bm}
\usepackage{caption}
\usepackage{color}
\usepackage{float}
\usepackage{graphicx}
\usepackage{listings}
\usepackage{subfig}
\usepackage{wrapfig}
\usepackage{multirow}

% Titulinio aprašas
\university{Vilniaus universitetas}
\faculty{Matematikos ir informatikos fakultetas}
\department{Programų sistemų bakalauro studijų programa}
\papertype{Bakalauro baigiamasis darbas}
\title{Neuroninio tinklu pritaikymas automatizuotam programinės įrangos sistemų problemų sprendimui}
\titleineng{Automated software system issue solving using deep neural networks}
\author{Mantas Petrikas}
\reviewer{lekt. Irus Grinis}
\supervisor{dr. Vytautas Valaitis}
\date{Vilnius – \the\year}

% Nustatymai
% \setmainfont{Palemonas}   % Pakeisti teksto šriftą į Palemonas (turi būti įdiegtas sistemoje)
\bibliography{bibliografija}

\begin{document}
\maketitle


\sectionnonumnocontent{Santrauka}

% Šiame nagrinėjamos šilumos laidumo uždavinio lygiagretino galimybes naudojant centrinius ir grafinius procesorius.
% Darbo metu buvo implentuoti centrinius procesorius naudojantys nuoseklusis ir lygiagretusis algorimtai bei grafinius procesorius naudojantis algoritmas.
% Algoritmus įgyvendinančios programos buvo testuojamos Vilniaus Universiteto Matematikos ir Informatikos fakulteto Skaitmeninių tyrimų ir skaičiavimų centro paskirstytų skaičiavimų tinkle.
% Centrinius procesorius naudojanti lygiagretujį algoritmą įgyvendinanti programa buvo ištestuota naudojant 256 branduolius ir buvo gautas 98 kartų pagreitėjimas.
% Grafinius procesorius naudojanti lygiagrečioji programa buvo ištestuota naudojant NVIDIA Tesla V100 SXM2 grafinę plokštę ir buvo gautas 360 kartų pagreitėjimas lyginant su nuosekliuoju algorimtu.
% Nustatyta, kad grafinius procesorius naudojanti šilumos lygi sprendžianti programa naudoja mažiau elektros energijos resursų nei centrinius procesorius naudojanti programa.


\raktiniaizodziai{GPT}

\sectionnonumnocontent{Summary}

% This work examines the parallelization possibilities of the heat transfer problem using CPUs and GPUs.
% Serial and parallel algorithms using CPU and algorithm using GPU were implanted during the work.
% The program that implement the algorithms were tested in the distributed computing network of the Digital Research and Computation Center of the Faculty of Mathematics and Informatics of Vilnius University.
% A program implementing a parallel algorithm using CPUs was tested using 256 cores and speedup of 98 was observed .
% A parallel program using graphics processors was tested using an NVIDIA Tesla V100 SXM2 graphics card and the process was 360 times faster to the serial algorithm.
% Also, this work constitrutes that an application that solves heat equation uses a graphics processor uses less power than an application that uses a central processing unit.
\keywords{GPT}

\tableofcontents

\sectionnonum{Įvadas}

Bakalauriniame darbe tiriamos modernių natūralios kalbos apdorojimo modelių pritaikymo galimybės programinės įrangos sistemų problemų sprendimui. 
Dauguma modernių giliaisiais neoriniais tinklais paremtų natūralios kalbos apdorijo sistemų tokiu kaip Salesforce sukurtas CodeT5 \cite{wang2021codet5}, Microsoft CodeBERT \cite{feng2020codebert} ar OpenAI GPT-3 \cite{brown2020language} sugeba generuoti programinio kodo dalis,
tačiau tokių modelių pritaikymo galimybės spręsti sistemos lygio problemas nėra iki galo ištirtos.


Darbo tikslas - sukurti giliaisiais neuroniniais tinklais paremtą sistemą kuris pagal programinės sistemos problemos aprašymą sugebėtų identifikuoti kodo dalis kurias reikėtų pakeisti norint išspręsti problemą ir pateikti galimus kodo pakeitimus.

Uždaviniai:
\begin{enumerate}
    \item išanalizuoti ir įvertinti dabartinių kodo generavimo įrankių galimybės spęsti sistemos lygio problemas
    \item surinkti sistemos funkcionalumo pakeitimų aprašymų ir realizuotų kodo pakeitimų skirta modelio mokymui
    \item suprojektuoti sistemą, kuri priima sistemos kodą ir funkcionalumo aprašymą ir gražiną sąrašą failų kuriuos reikėtų pakeisti norint įgyvendinti funkcionalumą
    \item apmokyti suprojektuotos sistemos giliuosius neuroninius modelius naudojant surinktą duomenų rinkinį ir įvertinti modelių veikimą
    \item praplėsti sukurtos sistemos funkcionalumą pridedant kodo generavimo galimybę
    \item įvertinti sukurtą sistemą palyginant su kitomis koda generuojančiomis sistemos
\end{enumerate}


Darbo rezultatai:
\begin{itemize}
    \item sistemos funkcionalumo pakeitimų aprašymų ir realizuotų kodo pakeitimų duomenų rinkinys
    \item kuriamos sistemos architektūros aprašymas
    \item realizuota sistema sugebanti identifikuoti failus, kurios reikia pakeisti norint įgyvendinti sistemos funkcionalumo pakeitimą
    \item realizuota sistema generuojanti kodo pakeitimus reikalingus sistemos funkcionalumo pakeisti
    \item sukurtos sistemos giliųjų neuroninių tinklų modelių įvertinimo ataskaita
\end{itemize}

\section{Moderniausi teksto generavimo modeliai}

\subsection{Transformatorių modeliai}

\section{Modelių rezulatų vertinimas}

\section{Duomenų rinkiniai}
% Įvade apibūdinamas darbo tikslas, temos aktualumas ir siekiami rezultatai. Darbo įvadas neturi
% būti dėstymo santrauka. Įvado apimtis 1–2 puslapiai.



\

% Medžiagos darbo tema dėstymo skyriuose pateikiamos nagrinėjamos temos detalės: pradinė me-
% džiaga, jos analizės ir apdorojimo metodai, sprendimų įgyvendinimas, gautų rezultatų apibendrinimas.
% Šios dalies turinys labai priklauso nuo darbo temos. Kursiniame darbe analizuojama dalykinė sritis, jo
% rezultate formuluojamas bakalauro darbe sprendžiamas uždavinys. Referatas neatitinka kursiniams
% darbams keliamų reikalavimų. Dėstymo skyriai gali turėti poskyrius ir smulkesnes sudėtines dalis, kaip
% punktus ir papunkčius

% Medžiaga turi būti dėstoma aiškiai, pateikiant argumentus. Tekste dėstomas
% trečiuoju asmeniu, t.y. rašoma ne „aš manau“, bet „autorius mano“, „autoriaus
% nuomone“. Reikėtų vengti informacijos nesuteikiančių frazių, pvz., „...kaip jau
% buvo minėta...“, „...kaip visiems žinoma...“ ir pan., vengti grožinės
% literatūros ar publicistinio stiliaus, gausių metaforų ar panašių meninės
% išraiškos priemonių.

% Darbo uždaviniai:
% \begin{itemize}
%     \item implementuoti nuoseklųjį algoritmą, sprendžianti šilumos pasiskirstymo uždavinį
%     \item implementuoti nuseklu ir įvertinti šilumos laidumo uždavinio algoritmo pagreitėjimą naudojant centrinius procesorius
%     \item suprojektuoti ir implementuoti šilumos laidumo uždavinio sprendimo algorimtą, naudojantį grafinių procesorių resursus
%     \item įvertinti grafinius procesorius naudojančio algoritmo našumą ir praktiškumą lyginant su centrinius procesorius naudojančiu algoritmu
%     \item palyginti gautus rezultatus rezultatus su kitais panašiais problemas nagrinėjančių mokslinių darbų rezultatais
% \end{itemize}

% Darbo rezultatai:
% \begin{itemize}
%     \item Nuoskelioji šilumos laidumo uždavinio sprendimo implementacija
%     \item Lygiagrečioji šilumos laidumo uždavinio algoritmo sprendimo naudojanti centrinius procesorius
%     \item Lygiagrečioji laidumo uždavinio sprendimo implementacija naudojanti grafinius procesorius
%     \item Algoritmų teorinių ir praktinių pagreitėjimų analyzė
%     \item Grafinius ir centrinius procesorius naudojančių algoritmų pagreitėjimų palyginimas
% \end{itemize}


% kuriame pateikiami tyrimo objektas ir aktualumas, darbo tikslas, keliami už-
% daviniai ir laukiami rezultatai, tyrimo metodai, numatomas darbo atlikimo procesas, apibūdi-
% nami darbui aktualūs literatūros šaltiniai.
% Pastaba. Darbo uždavinyje apibrėžiamas siekiamas rezultatas, kad būtų galimybė išmatuoti,
% ar tikslai ir uždaviniai yra išspręsti, bei kokiu lygiu (vertinant kiekybę bei kokybę). Pavyz-
% džiui, „Atlikti literatūros .... analizę“ nėra tinkamas uždavinys, nes nusako procesą, tačiau
% neapibrėžia jo rezultato. Tinkamos uždavinio formuluotės šablonai: „Išanalizuoti literatūrą
% ... siekiant apžvelgti ir įvertinti /... metodų tinkamumą sprendžiamam uždaviniui/privalu-
% mus ir trūkumus sprendžiant ... uždavinį/rekomenduojamas ... projektavimo gaires, šablonus
% ir pan.

% \sectionnonum{Įvadas}
% Įvade nurodomas darbo tikslas ir uždaviniai, kuriais bus įgyvendinamas tikslas,
% aprašomas temos aktualumas, apibrėžiamas tiriamasis objektas akcentuojant
% neapibrėžtumą, kuris bus išspręstas darbe, aptariamos teorinės darbo prielaidos
% bei metodika, apibūdinami su tema susiję literatūros ar kitokie šaltiniai,
% temos analizės tvarka, darbo atlikimo aplinkybės, pateikiama žinių apie
% naudojamus instrumentus (programas ir kt., jei darbe yra eksperimentinė dalis).
% Darbo įvadas neturi būti dėstymo santrauka. Įvado apimtis 2 -- 4 puslapiai.

% \section{Medžiagos darbo tema dėstymo skyriai}
% Medžiagos darbo tema dėstymo skyriuose išsamiai pateikiamos nagrinėjamos temos
% detalės: pradiniai duomenys, jų analizės ir apdorojimo metodai, sprendimų
% įgyvendinimas, gautų rezultatų apibendrinimas.


% Skyriai gali turėti poskyrius ir smulkesnes sudėtines dalis, kaip punktus ir
% papunkčius.




% \sectionnonum{Rezultatai ir išvados}
% Rezultatų ir išvadų dalyje turi būti aiškiai išdėstomi pagrindiniai darbo rezultatai (kažkas išana-
% lizuota, kažkas sukurta, kažkas įdiegta) ir pateikiamos išvados (daromi nagrinėtų problemų sprendimo
% metodų palyginimai, teikiamos rekomendacijos, akcentuojamos naujovės).

\printbibliography[heading=bibintoc]  % Šaltinių sąraše nurodoma panaudota
% literatūra, kitokie šaltiniai. Abėcėlės tvarka išdėstomi darbe panaudotų
% (cituotų, perfrazuotų ar bent paminėtų) mokslo leidinių, kitokių publikacijų
% bibliografiniai aprašai. Šaltinių sąrašas spausdinamas iš naujo puslapio.
% Aprašai pateikiami netransliteruoti. Šaltinių sąraše negali būti tokių
% šaltinių, kurie nebuvo paminėti tekste. Šaltinių sąraše rekomenduojame
% necituoti savo kursinio darbo, nes tai nėra oficialus literatūros šaltinis.
% Jei tokių nuorodų reikia, pateikti jas tekste.

% \sectionnonum{Sąvokų apibrėžimai}
% \sectionnonum{Santrumpos}
% Sąvokų apibrėžimai ir santrumpų sąrašas sudaromas tada, kai darbo tekste
% vartojami specialūs paaiškinimo reikalaujantys terminai ir rečiau sutinkamos
% santrumpos.

\end{document}
