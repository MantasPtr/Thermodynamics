\section{Literatūros ir esamų metodų apžvalga}

Siame skyriuje apžvalgiama su sesiju valdymu susijusi literatūra



\section{Sesiju panaudojamas}

Sesijos saugo vartotojo informacija apie dabartine saveikos busena su programa. Tai naudojama internetinese programose, 

Įprasta internetine programa išlaiko kiekvieno prisijungusio vartotojo sesiją tol, kol vartotojas yra prisijungęs. 
Sesijos būsena yra tai, kaip programos prisimena vartotojo tapatybę, suasmeninimo informaciją, prisijungimo duomenis, ar
naujausius veiksmus. Sesijoje kaupiami duomenys leidzia greiciau pasiekti reikiamus duomenis, daznai 

Sesijos gyvavimo laikas pasibaigia, kai vartojas specifiškai iššaukia atsijungimo veiksmą 
arba tam tikra laiko tarpą kvieciami su sesija susije veiksmai.
Priklausimai nuo sesijos paskirties, kai kurie duomenys gali būti išsaugoti duomenų bazėje, vėlesniam naudojimui, 
arba trumpalaikės informacija gali būti sunaikinama pasibaigus sesijai.

Sesijos būsena yra panaši į spartinančiąją atmintį (ang. cache), skiriasi tik duomenų valdymo modelis.
Spartinančioji atmintis yra tolerantiška duomenų praradimui ir ją bet kada galima atkurti iš pirminės duomenų bazės, 
tačiau atnaujinant spartinančios atminties duomenis reikia atnaujinti ir pagrindinę duomenų talpyklą.
Sesijos duomenis sukuriami prasidedant sesijai, ir vėliau nėra garantuojama kad sesijos duomenys nepasikeis t.y. juos bus galima atkurti duomenų šaltinio.
Sesijos duomenys dažniausiai išsaugomi į pagrindinę talpyklą tik pasibaigus sesijai.
Sesijos duomenys gali būti laikini arba nuolatiniai - t.y.  pasibaigus vartotojo sesijai duomenys gali būti sunaikinami arba išsaugoti vėliasniam naudojimui..


\section{Resursu valdymas}

Norinti sumažinti programos prieeinamumo laikotarpį, 



\section{Komunikacijos klaidos}

Duomenų perdavimo sutrikimai gali atsitiki dėl per didelės tinklo apkrovos, aptarnaujančios serverio perkrovos,
techninės ar programinės įrangos klaidos, paslaugų teikimą trikdančių (angl. denial-of-service) atakų \cite{ayari2008fault}.


Session pool management

Resource management
Redudancy
Resource sharing
Session management
Network faults
Net split
False negatives

t56