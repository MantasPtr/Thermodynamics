\section{Algortimo analizė}

Šiame skyriuje aprašomas darbe nagrinėjamas asynchroniško sesiju valdymo algoritmo \cite{petrauskas2018asynchronous} modelis, pagrindiniai algoritmo veikimo principai, daromos prielaidos, galimi
modelio ir realios sistemos neatitikimai.

Modelis sudarytas iš 3 sistemų grupių. Kientų-vartotojų (angl. consumers), 
 kurie atitinka siekiančių gauti informacija ar laikiantys tam tikro rezultato, 
 tiekėjų (angl. providers), kurie aptarnauja kientus teikdami duomenis ar atlikdami funkcijas. 
Bendravimas tarp vartotojų ir tiekėjų vyksta sesijų pagalba ir koordinatorių (angl. coordinators), 
 valdančių vartotojų ir tiekėjų komunikaciją. 
Tiek vartotojų, tiek vartojų grupėje esantys mazgai (angl. nodes) patys nežino 
 apie kitų to pačio tipo mazgų egzistavimą, ir neturi valdančio proceso (angl. master node), 
 kuris galėtų valdyti bendravimą tarp jų.

Tinklo topologija modelyje nėra detalizuojama,
 tačiau daroma prielaida kad visi koordinuojantys mazgai žino apie visus kitus koordinuojančius mazgus,
 o sesijos gali sudaryti ryšį su bet kuriuo tiekėjo mazgu.
Tai nebūtinai atitinka realiaus panaudojimo atvejus, 
 nes paskirtų sistemų dalys gali būti veikti skirtinguose pasaulio vietose 
 ir nebutinai būti vienodai pasiekiamos skirtuose tinkluose (ang. subnets)
 tačiau sistemos topologijos problematika nėra nagrinėjama šiame darbe.


\subsection{Modelio būsena}

Modelis aprašomas TLA\textsuperscript{+} modelio kalba, 
 ir jo savybių tikrinui naudojamas TLA+ toolbox įrankis,
 todėl norint patikriniti modelio teisingumą reikia apriboti galimų tinklo mazgų kombinacijų skaičių. 
Tai daroma keičiant modelio konfiguracija, keičiant sesijų, tiekėjų ir vartotojų skaičių 
 prieš atliekant formalų modelio tikrinimą. 
Šie parametrai nesikeičia modelio tikrinimo metu,
 tačiau gali būti pakeisti prieš paleidžiant skirtingas simuliacijas.  
Modelyje laikoma kad kiekvienas vartotojas gali turi vienoda sesijų kiekį,
 tačiau sesija nebūtinai gali būti prisijungusi prie vartotojo.
Kiekvienas vartotojo mazgas turi savo koordinatorių.


\subsection{Tiekėjų būsenos pakeitimai}

Modelyje tiekėjai turi 2 būsenų perėjimus.
Jei tiekėjas veikia, jis gali tapti neveikiančiu ir atvirščiai- neveikiantis gali tampti veikiančiu.
Sąlygos dėl kurių tiekėjai gali nustoti veikti ar vėl pradėti veikti nėra apibrėžtos ir modelyje nemodeliuojamos.
Tiekėjai patys nepraneša aptarnaujanties mazgams apie savo būsenos pakeitimus, 
žinučių apie klientų būsenų pasikeitimus sekimu rupinasi prie tiekėjo prisijungę vartotojai.

\subsection{Vartojų būsenos pakeitimai}

Modelyje vartotai turi 3 būsenų pakeitimus: 
vartotojo sesijos prisijungimą prie veikiančio tiekėjo, 
sesijos atsijungimą gavus žinutę iš koordinatoriaus apie neveikiantį tiekėją,
sesijos atsijungimą nustojus veikti tiekėjui.

Komunikacija tarp tiekėjo ir vartotojo šiame algortimo modelyje nėra modeliuojama, 
 nes normalus sistemos veikimas nekeičia sistemos būsenos, 
 nes laikoma kad klientas yra visada lieka prisijungęs prie to pačio tiekėjo. 


\subsection{Koordinatoriaus būsenos pakeitimai}

Modelyje koordinatoriai turi 3 būsenų pakeitimus: 
žinučių iš sesijų apie neaktyvius tiekėjus apdoriją,
pranešimą kitiems koordinatoriams apie neveikiančius tiekėju mazgus,
tikrinimą ar tiekėjai tapo aktyviais.

Koordinuojantis mazgas, gavęs pranešimą apie neveikiantį tiekėją,
 išsisaugo savo busenoje tiekėja kaip neveikiantį 
 ir asynchroniškai praneša visoms savo valdomoms sesijoms apie neveikiantį mazgą.

Koordinatorius žinantis apie neveikiantį mazga, 
 nuolatos asynchroniškai tikrinina tiekėjo būsena ir
 sužinojęs apie pradėjusi veikti mazgą, atnaujina savo būseną,
 pažyminti tiekėją kaip veikiantį.

Klientai, iš koordinuojančios gavę žinutę apie neveikiantį tiekėjo mazgą,
 kiekvienai sesijai atskirai asynchronišku būdu nurodo atsijungti nuo kliento.
Atsijungusios sesijos gali prisijungti prie kito veikiančio tiekėjo.
Modelyje apraščiame darbe \cite{petrauskas2018asynchronous} daroma prielaida, 
 kad sesijoje vieno tiekėjo pakeimas kitu yra leidžiamą operacija, 
 nors pabrėžiama, kad realiame scenarijuje tiekėjo mazgo pakeitimas gali 
 pasireikšti sistemos veikimo suletėjimu ar nenuoseklių (ang. inconsistent) duomenų atsiradimu, 
 todėl šios operacija turėtų būti kaip įmanoma labiau vengiama.

