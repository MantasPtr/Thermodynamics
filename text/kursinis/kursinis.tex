\documentclass{VUMIFPSbakalaurinis}
\usepackage{algorithmicx}
\usepackage{algorithm}
\usepackage{algpseudocode}
\usepackage{amsfonts}
\usepackage{amsmath}
\usepackage{bm}
\usepackage{caption}
\usepackage{color}
\usepackage{float}
\usepackage{graphicx}
\usepackage{listings}
\usepackage{subfig}
\usepackage{wrapfig}

% Titulinio aprašas
\university{Vilniaus universitetas}
\faculty{Matematikos ir informatikos fakultetas}
\department{Programų sistemų bakalauro studijų programa}
\papertype{Kursinis darbas}

\title{Šilumos laidumo uždavinio lygiagretinimas MIF klasteryje}
\titleineng{Parallelization of the steady-state heat problem on the MIF computing cluster}
\author{Mantas Petrikas}
\supervisor{dr. Rokas Astrauskas}
\date{Vilnius – \the\year}

% Nustatymai
% \setmainfont{Palemonas}   % Pakeisti teksto šriftą į Palemonas (turi būti įdiegtas sistemoje)
\bibliography{bibliografija}

\begin{document}
\maketitle



\tableofcontents


\sectionnonum{Įvadas}


% Įvade apibūdinamas darbo tikslas, temos aktualumas ir siekiami rezultatai. Darbo įvadas neturi
% būti dėstymo santrauka. Įvado apimtis 1–2 puslapiai.

Šiame darbe bus tiriamos šilumos lygties paralelizavimo galimybes naudojant centrinius(ang. CPU) procesorius. 
Šilumos uždavinys yra vienas iš Laplaso lygties pritaikymo galimybių. 
Šios antros eilės dalinės diferencialinės lygtys plačiai naudojamos fizikoje, sprendžiant elektrostatikos \cite{house2008analytic}, gravitacijos, magnetizmo \cite{blakely1996potential}, pastovios būsenos temperatūrų \cite {berntsson2001numerical} ir hidrodinamikos \cite{kadanoff1985simulating} problemas. 
Darbe narginėjami algoritmo palerizavimo teorinis ir praktinis pagreitėjimai pasitelkiant skirtingas duomenų padalimo branduoliams strategijas. 
Visi praktiniai eksperimentai buvo vykdomi naudojant Matemematikos ir Informatikos fakulteto Skaitmeninių tyrimų ir skaičiavimų centro paskirstytų skaičiavimų tinklo resursus.

\subsectionnonum{Darbo tikslas ir uždaviniai}

% Darbo tikslas - įvertinti ir palyginti šilumos laidumo uždavinio lygiagretinimo algoritmo efektyvumą naudojant centrinius ir grafinius procesorius.

% Uždaviniai:
% \begin{itemize}
%     \item implementuoti ir įvertinti šilumos laidumo uždavinio algoritmo pagreitėjimą naudojant centrinius procesorius
%     \item suprojektuoti ir implementuoti šilumos laidumo uždavinio sprendimo algorimtą, naudojantį grafinių procesorių resursus
%     \item įvertinti grafinius procesorius naudojančio algoritmo našumą ir praktiškumą lyginant su centrinius procesorius naudojančiu algoritmu
%     \item palyginti gautus rezultatus rezultatus su kitais panašiais problemas nagrinėjančių mokslinių darbų rezultatais
% \end{itemize}

% \subsectionnonum{Laukiami rezultatai}
% \begin{itemize}
%     \item Šilumos laidumo uždavinio algoritmo implementacija naudojanti centrinius procesorius
%     \item Šilumos laidumo uždavinio sprendimo implementacija naudojanti grafinius procesorius
%     \item Algoritmų teorinių ir praktinių pagreitėjimų analyzė
%     \item Grafinius ir centrinius procesorius naudojančių algoritmų pagreitėjimo palyginimas
% \end{itemize}
% \subsectionnonum{Darbo metodai}
% \begin{itemize}
%     \item Algoritmų veikimo laiko matavimas, keičiant jo parametrus MIF klasteriuose
%     \item Algoritmų praktinių ir teorinių pagreitėjimų palyginomiji analyzė
%     \item Susijusios literatūros analyzė
% \end{itemize}

% Šiame darbe nagrinėjama šilumos lygties uždavinio algoritmo paralelizavimo galimybės, naudojant MIF klasteryje esančius resursus. 
% Šilumos lygties uždavinys šio darbo ribose apibrėžiamas kaip dvimačio objekto temperatūros


Uždaviniai:
\begin{itemize}
    \item implementuoti nuoseklųjį šilumos laidumo uždavinio algoritmą
    \item implementuoti lygiagretūjį šilumos laidumo uždavinio algoritmą pagreitėjimą naudojant centrinius procesorius
    \item suprojektuoti ir implementuoti šilumos laidumo uždavinio sprendimo algorimtą, naudojantį grafinių procesorių resursus
    \item įvertinti grafinius procesorius naudojančio algoritmo našumą ir praktiškumą lyginant su centrinius procesorius naudojančiu algoritmu
    \item palyginti gautus rezultatus rezultatus su kitais panašiais problemas nagrinėjančių mokslinių darbų rezultatais
\end{itemize}

\section{Medžiagos darbo tema dėstymo skyriai}
% Medžiagos darbo tema dėstymo skyriuose pateikiamos nagrinėjamos temos detalės: pradinė me-
% džiaga, jos analizės ir apdorojimo metodai, sprendimų įgyvendinimas, gautų rezultatų apibendrinimas.
% Šios dalies turinys labai priklauso nuo darbo temos. Kursiniame darbe analizuojama dalykinė sritis, jo
% rezultate formuluojamas bakalauro darbe sprendžiamas uždavinys. Referatas neatitinka kursiniams
% darbams keliamų reikalavimų. Dėstymo skyriai gali turėti poskyrius ir smulkesnes sudėtines dalis, kaip
% punktus ir papunkčius

% Medžiaga turi būti dėstoma aiškiai, pateikiant argumentus. Tekste dėstomas
% trečiuoju asmeniu, t.y. rašoma ne „aš manau“, bet „autorius mano“, „autoriaus
% nuomone“. Reikėtų vengti informacijos nesuteikiančių frazių, pvz., „...kaip jau
% buvo minėta...“, „...kaip visiems žinoma...“ ir pan., vengti grožinės
% literatūros ar publicistinio stiliaus, gausių metaforų ar panašių meninės
% išraiškos priemonių.


\sectionnonum{Rezultatai ir išvados}

% \subsection{Poskyris}
% Citavimo pavyzdžiai: cituojamas vienas šaltinis \cite{PvzStraipsnLt}; cituojami
% keli šaltiniai \cite{PvzStraipsnEn, PvzKonfLt, PvzKonfEn, PvzKnygLt, PvzKnygEn,
% PvzElPubLt, PvzElPubEn, PvzMagistrLt, PvzPhdEn}.

% \subsubsection{Skirsnis}
% \subsubsubsection{Straipsnis}
% \subsubsection{Skirsnis}
% \section{Skyrius}
% \subsection{Poskyris}
% \subsection{Poskyris}



% kuriame pateikiami tyrimo objektas ir aktualumas, darbo tikslas, keliami už-
% daviniai ir laukiami rezultatai, tyrimo metodai, numatomas darbo atlikimo procesas, apibūdi-
% nami darbui aktualūs literatūros šaltiniai.
% Pastaba. Darbo uždavinyje apibrėžiamas siekiamas rezultatas, kad būtų galimybė išmatuoti,
% ar tikslai ir uždaviniai yra išspręsti, bei kokiu lygiu (vertinant kiekybę bei kokybę). Pavyz-
% džiui, „Atlikti literatūros .... analizę“ nėra tinkamas uždavinys, nes nusako procesą, tačiau
% neapibrėžia jo rezultato. Tinkamos uždavinio formuluotės šablonai: „Išanalizuoti literatūrą
% ... siekiant apžvelgti ir įvertinti /... metodų tinkamumą sprendžiamam uždaviniui/privalu-
% mus ir trūkumus sprendžiant ... uždavinį/rekomenduojamas ... projektavimo gaires, šablonus
% ir pan.

% \sectionnonum{Įvadas}
% Įvade nurodomas darbo tikslas ir uždaviniai, kuriais bus įgyvendinamas tikslas,
% aprašomas temos aktualumas, apibrėžiamas tiriamasis objektas akcentuojant
% neapibrėžtumą, kuris bus išspręstas darbe, aptariamos teorinės darbo prielaidos
% bei metodika, apibūdinami su tema susiję literatūros ar kitokie šaltiniai,
% temos analizės tvarka, darbo atlikimo aplinkybės, pateikiama žinių apie
% naudojamus instrumentus (programas ir kt., jei darbe yra eksperimentinė dalis).
% Darbo įvadas neturi būti dėstymo santrauka. Įvado apimtis 2 -- 4 puslapiai.

% \section{Medžiagos darbo tema dėstymo skyriai}
% Medžiagos darbo tema dėstymo skyriuose išsamiai pateikiamos nagrinėjamos temos
% detalės: pradiniai duomenys, jų analizės ir apdorojimo metodai, sprendimų
% įgyvendinimas, gautų rezultatų apibendrinimas.


% Skyriai gali turėti poskyrius ir smulkesnes sudėtines dalis, kaip punktus ir
% papunkčius.




% \sectionnonum{Rezultatai ir išvados}
% Rezultatų ir išvadų dalyje turi būti aiškiai išdėstomi pagrindiniai darbo rezultatai (kažkas išana-
% lizuota, kažkas sukurta, kažkas įdiegta) ir pateikiamos išvados (daromi nagrinėtų problemų sprendimo
% metodų palyginimai, teikiamos rekomendacijos, akcentuojamos naujovės).

\printbibliography[heading=bibintoc]  % Šaltinių sąraše nurodoma panaudota
% literatūra, kitokie šaltiniai. Abėcėlės tvarka išdėstomi darbe panaudotų
% (cituotų, perfrazuotų ar bent paminėtų) mokslo leidinių, kitokių publikacijų
% bibliografiniai aprašai. Šaltinių sąrašas spausdinamas iš naujo puslapio.
% Aprašai pateikiami netransliteruoti. Šaltinių sąraše negali būti tokių
% šaltinių, kurie nebuvo paminėti tekste. Šaltinių sąraše rekomenduojame
% necituoti savo kursinio darbo, nes tai nėra oficialus literatūros šaltinis.
% Jei tokių nuorodų reikia, pateikti jas tekste.

% \sectionnonum{Sąvokų apibrėžimai}
% \sectionnonum{Santrumpos}
% Sąvokų apibrėžimai ir santrumpų sąrašas sudaromas tada, kai darbo tekste
% vartojami specialūs paaiškinimo reikalaujantys terminai ir rečiau sutinkamos
% santrumpos.

\end{document}
