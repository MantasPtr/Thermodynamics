\documentclass{VUMIFPSbakalaurinis}
\usepackage{algorithmicx}
\usepackage{algorithm}
\usepackage{algpseudocode}
\usepackage{amsfonts}
\usepackage{amsmath}
\usepackage{bm}
\usepackage{caption}
\usepackage{color}
\usepackage{float}
\usepackage{graphicx}
\usepackage{listings}
\usepackage{subfig}
\usepackage{wrapfig}

% Titulinio aprašas
\university{Vilniaus universitetas}
\faculty{Matematikos ir informatikos fakultetas}
% \institute{Informatikos institutas}  % Užkomentavus šią eilutę - institutas neįtraukiamas į titulinį
\department{Programų sistemų bakalauro studijų programa}
\papertype{Bakalauro baigiamojo darbo planas}

\title{Neuroninio tinklu pritaikymas automatizuotam programinės įrangos sistemų problemų sprendimui}

\titleineng{Automated software system issue solving using deep neural networks}
\author{Mantas Petrikas}
% \secondauthor{Vardonis Pavardonis}   % Pridėti antrą autorių
\supervisor{dr. Vytautas Valaitis}
% \reviewer{doc. dr. Vardauskas Pavardauskas}
\date{Vilnius – \the\year}

% Nustatymai
% \setmainfont{Palemonas}   % Pakeisti teksto šriftą į Palemonas (turi būti įdiegtas sistemoje)
\bibliography{bibliografija}

\begin{document}
\maketitle


\sectionnonum{Darbo planas}

Bakalauriniame darbe tiriamos modernių natūralios kalbos apdorojimo modelių pritaikymo galimybės programinės įrangos sistemų problemų sprendimui. 
Dauguma modernių giliaisiais neoriniais tinklais paremtų natūralios kalbos apdorijų sistemų tokiu kaip Salesforce sukurtas CodeT5 \cite{wang2021codet5}, Microsoft CodeBERT \cite{feng2020codebert} ar OpenAI GPT-3 \cite{brown2020language} sugeba labai gerai generuoti kodo dalis,
tačiau tokių modelių pritaikymo galimybės spręsti sistemos lygio problemas nėra iki galo ištirtos.

\subsectionnonum{Darbo tikslas ir uždaviniai}

Darbo tikslas - sukurti giliaisiais neuroniniais tinklais paremtą sistemą kuris pagal programinės sistemos problemos aprašymą sugebėtų identifikuoti kodo dalis kurias reikėtų pakeisti norint išspręsti problemą ir pateikti galimus kodo pakeitimus.

Uždaviniai:
\begin{enumerate}
    \item išanalizuoti ir įvertinti dabartinių kodo generavimo įrankių galimybės spęsti sistemos lygio problemas
    \item surinkti sistemos funkcionalumo pakeitimų aprašymų ir realizuotų kodo pakeitimų skirta modelio mokymui
    \item suprojektuoti sistemą, kuri priima sistemos kodą ir funkcionalumo aprašymą ir gražiną sąrašą failų kuriuos reikėtų pakeisti norint įgyvendinti funkcionalumą
    \item apmokyti suprojektuotos sistemos giliuosius neuroninius modelius naudojant surinktą duomenų rinkinį ir įvertinti modelių veikimą
    \item praplėsti sukurtos sistemos funkcionalumą pridedant kodo generavimo galimybę 
    \item įvertinti sukurtą sistemą palyginant su kitomis koda generuojančiomis sistemos
\end{enumerate}

\subsectionnonum{Laukiami rezultatai}
\begin{enumerate}
    \item sistemos funkcionalumo pakeitimų aprašymų ir realizuotų kodo pakeitimų duomenų rinkinys
    \item kuriamos sistemos architektūros aprašymas
    \item realizuota sistema sugebanti identifikuoti failus, kurios reikia pakeisti norint įgyvendinti sistemos funkcionalumo pakeitimą
    \item realizuota sistema generuojanti kodo pakeitimus reikalingus sistemos funkcionalumo pakeisti
    \item sukurtos sistemos giliųjų neuroninių tinklų modelių įvertinimo ataskaita 

\end{enumerate}

\subsectionnonum{Darbo metodai}
\begin{itemize}
    \item Susijusios literatūros, duomenų rinkinių ir giliųjų neorinių tinklų modelių analizė
    \item Skirtingu kodo generavimo modelių palyginamoji analizė
    \item Atvirai pasiekiamų funkcionalumo pakeitimų aprašymų ir realizuotų kodo pakeitimų šaltinių analizė
    \item Neuroninių tinklų pritaikymas problemai spęsti naudojant perkeliamąjį  mokymasi

\end{itemize}

\subsectionnonum{Darbo eiga}

Darbe bus įvertinami esami atviro ir uždaro kodo produktų galimybės generuoti sistemos pakeitimus spręsti, pateikiant naujo funkcionalumo aprašymą.
Darbe bus palyginamos skirtingos giliųjų neuroninių tinklų architektūros naudojamos šiai užduočiai spęsti.
Apžvelgiami skirtingi duomenų rinkiniai naudoti apmokyti giliųjų neuroniniams tinklams ir jų aktualumas tiriamai problemai.
Darbe bus įvertinos skirtingi duomenų apdorojimo ir tokenizavimo technikos norint paruošti duomenys modelio mokymui.
Taip pat apžvelgiami modelio mokymui ir validavimui naudojami kriterijai.


% šilumos laidumo uždavinys, jo pritaikymo ir praplėtimo galimybes.
% Darbe bus nagrinėjami CPU ir GPU architektūriniai skirtumai, siekiant apžvelgti ir įvertinti technologijų pritaikomomumą, privalumus ir trūkumus spendžiant šilumos pasiskirstymo pastovios būsenos sistemose (ang. steady-state heat distribution) uždavinį.
% Darbo metu bus nagrinėjamas ir įgyvendinamas šilumos lygties sprendimas apribotoje dvimatinėjė erdvėje, kurioje kraštų temperatūros yra žinomos ir nekintančios.
% Darbe bus apžvelgiami vertimo kriterijai, kurias galima įvertinti ir palyginti algoritmo našumą, praktines taikymo galimybes ir kaštus. Bus įvedami kriterijai reikalingi palyginti algoritmo našumą naudojant centrinius ir grafinius procesorius.
% Taip pat bus apžvelgiamas vienodos atminties prieigos (ang. UMA - uniform memory access) \cite{rogers2013amd} \cite{bader2001using} ir nevienodos atminties prieiga (ang. NUMA - non uniform memory access) \cite{lameter2013overview} programų architektūra, įvertinant jų trūkumus ir privalumus. 



Darbo praktinėje dalyje naudojantis laisvais prieinamais šaltiniais ir duomenų rinkiniais bus atrinktas duomenų rinkinys tinkamas šio darbo nagrinėjamos problemos sprendimui naudojamo giliojo neutroninio tinklo mokymui.
Duomenų rinkinį turėtų sudaryti sistemos kodas, sistemoje esančios problemos ar naujo funkcionalumo aprašas, ir kodo pakeitimai sistemos funkcionalumo pakeitimui įgyvendinti.
Norint pagreitinti sistemos sukūrimo ir apmokymo laiką bus bandoma kuo labiau pritaikyti jau laisvai prieinamus apmokytus modelius, tokius kaip CodeBERT \cite{feng2020codebert} ar GPT-2 \cite{radford2019language}.
Turint duomenų rinkinį bus atrinka užduočiai tinkama teksto tokenizavimui skirta sistema.
Tada bus sukurta sistemos architetūra ir apmokytas gilusis neoroninis tinklas sugebentis identifikuoti kurios sistemos failus reikia pakeisti norint pakeisti sistemos funkcionalumą.
Realizavus šią sistemą, atsižvelgiant į darbui pateikti likusį laiką, bus bandomą ją praplėsti galimybe generuoti kodo pakeitimus reikalingus problemos sprendimui.

% Darbo praktinėjė dalyje bus įgyvendimas nuoseklus algoritmas naudojant 1 centrinį procesorių, įvertinamas algoritmo našumas. 
% Taip pat bus įgyvendimas paralelizuotas šilumos lygties spendimo algoritmas, naudojant NUMA architektūra, įvertimas algortimo teorinis ir praktinis pagreitėjimas testuojant algoritmą MIF klasteryje keičiant programos parametrus ir procesorių skaičių.
% Apžvelgiami lygiagretino technologijos ir algortimo pakeitimai reikalingi paleisti algoritmą naudojant grafinius procesorius. Įgyvendimas lygiagretus algoritmas naudojant 1 grafinį procesorių. Įvertimas algoritmo pagreitėjimas, keičiant algoritmo parametrus. 
% Palyginimas algoritmo veikimas naudojant centrinis ir grafinius procesorius. Apžvelgiamos lygiagretinimo algoritmo galimybės naudojant kelis grafinius procesorius, su galimybę įgyvendinti ir ištirti algoritmo našumą naudojant kelis grafinius procesorius MIF klasteryje.





% kuriame pateikiami tyrimo objektas ir aktualumas, darbo tikslas, keliami už-
% daviniai ir laukiami rezultatai, tyrimo metodai, numatomas darbo atlikimo procesas, apibūdi-
% nami darbui aktualūs literatūros šaltiniai.
% Pastaba. Darbo uždavinyje apibrėžiamas siekiamas rezultatas, kad būtų galimybė išmatuoti,
% ar tikslai ir uždaviniai yra išspręsti, bei kokiu lygiu (vertinant kiekybę bei kokybę). Pavyz-
% džiui, „Atlikti literatūros .... analizę“ nėra tinkamas uždavinys, nes nusako procesą, tačiau
% neapibrėžia jo rezultato. Tinkamos uždavinio formuluotės šablonai: „Išanalizuoti literatūrą
% ... siekiant apžvelgti ir įvertinti /... metodų tinkamumą sprendžiamam uždaviniui/privalu-
% mus ir trūkumus sprendžiant ... uždavinį/rekomenduojamas ... projektavimo gaires, šablonus
% ir pan.

% \sectionnonum{Įvadas}
% Įvade nurodomas darbo tikslas ir uždaviniai, kuriais bus įgyvendinamas tikslas,
% aprašomas temos aktualumas, apibrėžiamas tiriamasis objektas akcentuojant
% neapibrėžtumą, kuris bus išspręstas darbe, aptariamos teorinės darbo prielaidos
% bei metodika, apibūdinami su tema susiję literatūros ar kitokie šaltiniai,
% temos analizės tvarka, darbo atlikimo aplinkybės, pateikiama žinių apie
% naudojamus instrumentus (programas ir kt., jei darbe yra eksperimentinė dalis).
% Darbo įvadas neturi būti dėstymo santrauka. Įvado apimtis 2 -- 4 puslapiai.

% \section{Medžiagos darbo tema dėstymo skyriai}
% Medžiagos darbo tema dėstymo skyriuose išsamiai pateikiamos nagrinėjamos temos
% detalės: pradiniai duomenys, jų analizės ir apdorojimo metodai, sprendimų
% įgyvendinimas, gautų rezultatų apibendrinimas.

% Medžiaga turi būti dėstoma aiškiai, pateikiant argumentus. Tekste dėstomas
% trečiuoju asmeniu, t.y. rašoma ne „aš manau“, bet „autorius mano“, „autoriaus
% nuomone“. Reikėtų vengti informacijos nesuteikiančių frazių, pvz., „...kaip jau
% buvo minėta...“, „...kaip visiems žinoma...“ ir pan., vengti grožinės
% literatūros ar publicistinio stiliaus, gausių metaforų ar panašių meninės
% išraiškos priemonių.

% Skyriai gali turėti poskyrius ir smulkesnes sudėtines dalis, kaip punktus ir
% papunkčius.

% \subsection{Poskyris}
% Citavimo pavyzdžiai: cituojamas vienas šaltinis \cite{PvzStraipsnLt}; cituojami
% keli šaltiniai \cite{PvzStraipsnEn, PvzKonfLt, PvzKonfEn, PvzKnygLt, PvzKnygEn,
% PvzElPubLt, PvzElPubEn, PvzMagistrLt, PvzPhdEn}.

% \subsubsection{Skirsnis}
% \subsubsubsection{Straipsnis}
% \subsubsection{Skirsnis}
% \section{Skyrius}
% \subsection{Poskyris}
% \subsection{Poskyris}

% \sectionnonum{Rezultatai ir išvados}
% Rezultatų ir išvadų dalyje išdėstomi pagrindiniai darbo rezultatai (kažkas
% išanalizuota, kažkas sukurta, kažkas įdiegta), toliau pateikiamos išvados
% (daromi nagrinėtų problemų sprendimo metodų palyginimai, siūlomos
% rekomendacijos, akcentuojamos naujovės). Rezultatai ir išvados pateikiami
% sunumeruotų (gali būti hierarchiniai) sąrašų pavidalu. Darbo rezultatai turi
% atitikti darbo tikslą.

\printbibliography[heading=bibintoc]  % Šaltinių sąraše nurodoma panaudota
% literatūra, kitokie šaltiniai. Abėcėlės tvarka išdėstomi darbe panaudotų
% (cituotų, perfrazuotų ar bent paminėtų) mokslo leidinių, kitokių publikacijų
% bibliografiniai aprašai. Šaltinių sąrašas spausdinamas iš naujo puslapio.
% Aprašai pateikiami netransliteruoti. Šaltinių sąraše negali būti tokių
% šaltinių, kurie nebuvo paminėti tekste. Šaltinių sąraše rekomenduojame
% necituoti savo kursinio darbo, nes tai nėra oficialus literatūros šaltinis.
% Jei tokių nuorodų reikia, pateikti jas tekste.

% \sectionnonum{Sąvokų apibrėžimai}
% \sectionnonum{Santrumpos}
% Sąvokų apibrėžimai ir santrumpų sąrašas sudaromas tada, kai darbo tekste
% vartojami specialūs paaiškinimo reikalaujantys terminai ir rečiau sutinkamos
% santrumpos.

\end{document}
