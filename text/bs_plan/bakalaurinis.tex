\documentclass{VUMIFPSbakalaurinis}
\usepackage{algorithmicx}
\usepackage{algorithm}
\usepackage{algpseudocode}
\usepackage{amsfonts}
\usepackage{amsmath}
\usepackage{bm}
\usepackage{caption}
\usepackage{color}
\usepackage{float}
\usepackage{graphicx}
\usepackage{listings}
\usepackage{subfig}
\usepackage{wrapfig}

% Titulinio aprašas
\university{Vilniaus universitetas}
\faculty{Matematikos ir informatikos fakultetas}
% \institute{Informatikos institutas}  % Užkomentavus šią eilutę - institutas neįtraukiamas į titulinį
\department{Programų sistemų bakalauro studijų programa}
\papertype{Bakalauro baigiamojo darbo planas}

\title{Šilumos laidumo uždavinio lygiagretinimo tyrimas naudojant centrinius ir grafinius procesorius}
\titleineng{Steady-state heat equasion parallization analysis on CPU and GPU}
\author{Mantas Petrikas}
% \secondauthor{Vardonis Pavardonis}   % Pridėti antrą autorių
\supervisor{dr. Rokas Astrauskas}
% \reviewer{doc. dr. Vardauskas Pavardauskas}
\date{Vilnius – \the\year}

% Nustatymai
% \setmainfont{Palemonas}   % Pakeisti teksto šriftą į Palemonas (turi būti įdiegtas sistemoje)
\bibliography{bibliografija}

\begin{document}
\maketitle

%% Padėkų skyrius
% \sectionnonumnocontent{}
% \vspace{7cm}
% \begin{center}
%     Padėkos asmenims ir/ar organizacijoms
% \end{center}

% \sectionnonumnocontent{Santrauka}
% Glaustai aprašomas darbo turinys: pristatoma nagrinėta problema ir padarytos
% išvados. Santraukos apimtis ne didesnė nei 0,5 puslapio. Santraukų gale
% nurodomi darbo raktiniai žodžiai. 
% % Nurodomi iki 5 svarbiausių temos raktinių žodžių (terminų).
% Vienas terminas gali susidėti iš kelių žodžių.
% \raktiniaizodziai{raktinis žodis 1, raktinis žodis 2, raktinis žodis 3, raktinis žodis 4, raktinis žodis 5}   

% \sectionnonumnocontent{Summary}
% Santrauka anglų kalba. Santraukos apimtis ne didesnė nei 0,5 puslapio.
% \keywords{keyword 1, keyword 2, keyword 3, keyword 4, keyword 5}

% \tableofcontents


\sectionnonum{Darbo planas}

Bakalauriniame darbe bus tiriamos šilumos lygties paralelizavimo galimybes naudojant centrinius(ang. CPU) ir grafinių procesorius (ang. GPU).

\subsectionnonum{Darbo tikslas ir uždaviniai}

Darbo tikslas - įvertinti ir palyginti šilumos laidumo uždavinio algoritimo efektyvumą naudojant centrinius ir grafinius procesorius.

Uždaviniai:
\begin{itemize}
    \item suprojektuoti ir implementuoti šilomos uždavinio sprendimo algorimtą, naudojantį grafinių procesorių resursus
    \item įvertinti grafinius procesorius naudojančio algoritmo našumą ir praktiškumą palyginant su centrinius procesoriu naudojančiu algoritmu.
    \item Nustatyti algoritmo našumo pagerėjimą naudojant
\end{itemize}

\subsectionnonum{Darbo eiga}

Darbo teorinėjė dalyje bus atžvelgiamas šilumos laidumo uždavinys, jo pritaikymo ir praplėtimo galimybes.
Darbe bus atliekama literatūros analyzė, nagrinėjami CPU ir GPU architectūriniai skirtumai, siekiant apžvelgti ir įvertinti technologijų pritaikomomumą, privalumus ir trukūmus spendžiant šilumos pasiskirtymo patovios busenos (ang. steady-state heat distribution) uždavinį.
Darbo metu bus nagrinėjamas ir įgyvendimas šilumos lygties sprendimas apribotoje dvimatinėjė erdvėje, kurioje kraštų temperatūros yra žinomos ir nekintančios.
Darbe bus apžvelgiami vertimo kriterijai, kurias galima įvertinti ir palyginti algoritmo našumą, praktines taikymo galimybes ir kaštus. Bus įvedami kriterijai reikalinti palyginti algoritmo našumą naudojant centinius ir grafinius procesorius.
Taip pat bus apžvelgiamas vienodos atminties prieigos (ang. UMA - uniform memory access) ir nevienodos atminties prieiga (ang. NUMA - non uniform memory access) programų architetūra, įvertinant jų trūkumus ir privalumus. 

Darbo praktinėjė dalyje bus įgyvendimas nuoseklus algoritmas naudojant 1 centrinį procesorių, įvertinamas algoritmo našumas. 
Taip pat bus įgyvendimas paralelizuotas šilumos lygties spendimo algoritmas, įvertimas algortimo teorinis ir praktinis pagreitėjimas keičiant programos parametrus ir procesorių skaičių.
Apžvelgiami lygiagretino technologijos ir algortimo pakeitimai reikalingi paleisti algoritmą naudojant grafinius procesorius. Įgyvendimas lygiagretus algoritmas naudojant 1 grafinį procesorių. Įvertimas algoritmo pagreitėjimas, keičiant algoritmo parametrus. 
Palyginimas algorimto veikimas naudojant centrinis ir grafinius procesius. Apžvelgiamos lygiagretinimo algoritmo galimybės naudojant kelis grafinius procesorius, su galimybę įgyvendinti ir ištirti algoritmo našumą naudojant kelis grafinius procesorius.


% kuriame pateikiami tyrimo objektas ir aktualumas, darbo tikslas, keliami už-
% daviniai ir laukiami rezultatai, tyrimo metodai, numatomas darbo atlikimo procesas, apibūdi-
% nami darbui aktualūs literatūros šaltiniai.
% Pastaba. Darbo uždavinyje apibrėžiamas siekiamas rezultatas, kad būtų galimybė išmatuoti,
% ar tikslai ir uždaviniai yra išspręsti, bei kokiu lygiu (vertinant kiekybę bei kokybę). Pavyz-
% džiui, „Atlikti literatūros .... analizę“ nėra tinkamas uždavinys, nes nusako procesą, tačiau
% neapibrėžia jo rezultato. Tinkamos uždavinio formuluotės šablonai: „Išanalizuoti literatūrą
% ... siekiant apžvelgti ir įvertinti /... metodų tinkamumą sprendžiamam uždaviniui/privalu-
% mus ir trūkumus sprendžiant ... uždavinį/rekomenduojamas ... projektavimo gaires, šablonus
% ir pan.

% \sectionnonum{Įvadas}
% Įvade nurodomas darbo tikslas ir uždaviniai, kuriais bus įgyvendinamas tikslas,
% aprašomas temos aktualumas, apibrėžiamas tiriamasis objektas akcentuojant
% neapibrėžtumą, kuris bus išspręstas darbe, aptariamos teorinės darbo prielaidos
% bei metodika, apibūdinami su tema susiję literatūros ar kitokie šaltiniai,
% temos analizės tvarka, darbo atlikimo aplinkybės, pateikiama žinių apie
% naudojamus instrumentus (programas ir kt., jei darbe yra eksperimentinė dalis).
% Darbo įvadas neturi būti dėstymo santrauka. Įvado apimtis 2 -- 4 puslapiai.

% \section{Medžiagos darbo tema dėstymo skyriai}
% Medžiagos darbo tema dėstymo skyriuose išsamiai pateikiamos nagrinėjamos temos
% detalės: pradiniai duomenys, jų analizės ir apdorojimo metodai, sprendimų
% įgyvendinimas, gautų rezultatų apibendrinimas.

% Medžiaga turi būti dėstoma aiškiai, pateikiant argumentus. Tekste dėstomas
% trečiuoju asmeniu, t.y. rašoma ne „aš manau“, bet „autorius mano“, „autoriaus
% nuomone“. Reikėtų vengti informacijos nesuteikiančių frazių, pvz., „...kaip jau
% buvo minėta...“, „...kaip visiems žinoma...“ ir pan., vengti grožinės
% literatūros ar publicistinio stiliaus, gausių metaforų ar panašių meninės
% išraiškos priemonių.

% Skyriai gali turėti poskyrius ir smulkesnes sudėtines dalis, kaip punktus ir
% papunkčius.

% \subsection{Poskyris}
% Citavimo pavyzdžiai: cituojamas vienas šaltinis \cite{PvzStraipsnLt}; cituojami
% keli šaltiniai \cite{PvzStraipsnEn, PvzKonfLt, PvzKonfEn, PvzKnygLt, PvzKnygEn,
% PvzElPubLt, PvzElPubEn, PvzMagistrLt, PvzPhdEn}.

% \subsubsection{Skirsnis}
% \subsubsubsection{Straipsnis}
% \subsubsection{Skirsnis}
% \section{Skyrius}
% \subsection{Poskyris}
% \subsection{Poskyris}

% \sectionnonum{Rezultatai ir išvados}
% Rezultatų ir išvadų dalyje išdėstomi pagrindiniai darbo rezultatai (kažkas
% išanalizuota, kažkas sukurta, kažkas įdiegta), toliau pateikiamos išvados
% (daromi nagrinėtų problemų sprendimo metodų palyginimai, siūlomos
% rekomendacijos, akcentuojamos naujovės). Rezultatai ir išvados pateikiami
% sunumeruotų (gali būti hierarchiniai) sąrašų pavidalu. Darbo rezultatai turi
% atitikti darbo tikslą.

\printbibliography[heading=bibintoc]  % Šaltinių sąraše nurodoma panaudota
% literatūra, kitokie šaltiniai. Abėcėlės tvarka išdėstomi darbe panaudotų
% (cituotų, perfrazuotų ar bent paminėtų) mokslo leidinių, kitokių publikacijų
% bibliografiniai aprašai. Šaltinių sąrašas spausdinamas iš naujo puslapio.
% Aprašai pateikiami netransliteruoti. Šaltinių sąraše negali būti tokių
% šaltinių, kurie nebuvo paminėti tekste. Šaltinių sąraše rekomenduojame
% necituoti savo kursinio darbo, nes tai nėra oficialus literatūros šaltinis.
% Jei tokių nuorodų reikia, pateikti jas tekste.

% \sectionnonum{Sąvokų apibrėžimai}
% \sectionnonum{Santrumpos}
% Sąvokų apibrėžimai ir santrumpų sąrašas sudaromas tada, kai darbo tekste
% vartojami specialūs paaiškinimo reikalaujantys terminai ir rečiau sutinkamos
% santrumpos.

\end{document}
