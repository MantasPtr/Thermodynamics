\subsubsection*{Tyrimo objektas}

  Šiame darbe analizuojamas algortimas, skirtas asinchroniškai valdyti sesijų komunikaciją tarp skirtingu paskritytos sistemos grupių (angl. ,,clusters'') \cite{petrauskas2018asynchronous, petrauskas2019effectiveness}.
  Algoritmas aprašytas TLA\textsuperscript{+} specifikavimo kalba \cite{lamporttla+}, jo korektiškumas yra tikrinamas atviro kodo ,,TLA\textsuperscript{+} toolbox'' įrankiu  \cite{kuppe2019tla+}.

\subsubsection*{Darbo aktualumas}
  
  Moderios didelio praleidumą turinčios sistemos apdoroja tukstančius, ar net milijonus užklausų vienu metu, ir greitai ir patikimai gražinti duomenis.
  Kad efektyviai pasiekti rezultatus, sistemos projektuojamos numatant galimę plėsti jas didiant serverių skaičių ir paskirstant apkrovą tarp jų \cite{soni2016nginx}.
  
  Paskirstytos sistemos suteikia turi daug privalumų - 
  padidėja sistemos prieinamumas, sumažinamas nekritinių sistemos dalių poveikis visos sistemos veiklai,
  suteikiama galimybė didinti sistemos pralaidumą paralelizuojant ir horizontaliai plečiant sistemos resursus. 
  Tačiau paskirstytos sistemos turi ir minusų. Dauguma paskirstytų sistemų problemų susiję su sudėtingesne infrastrukūra ir diegimu (ang. deployment), 
  sistemos būsėnos valdymu ir  vientisumo išlaikymu visose sistemos dalyse \cite{thones2015microservices}.
  Norinti išlaikyti didelį sistemos pasiekiamumą ir operacijų greitumą, dalis sistemų naudoja asynchoninis bendravimą tarp skirtingų sistemų dalių, 
  užtikrinant tik galutinį būsenos nuoseklumą (ang. eventual consistency) \cite{vogels2009eventually}.


  TLA\textsuperscript{+} yra formali, predikatų logiką paremta, aukšto lygio modeliavo kalba, leidžianti projektuoti, aprašyti ir tikrinti sistemų korektiškumą \cite{kuppe2019tla}.
  Ši kalba yra naudojama didžiosiose kompanijose, ,,Intel'' \cite{batson2002high} ,,Amazon'' komponijoje \cite{newcombe2015amazon}, Microsoft ir kitose. 
  Specifikavimo kalba leidžia naudojant predikatų logiką aprašyti pradinę sistemos būseną,
  leistinus sistemos būsenos pakitimus ir savybės kurias modelis turi tenkinti, 
  tada naudojant ,,TLA toolbox'' įrankį" patikrinti ar visos sistemos būsenos tenkina pasirinktas savybes.
  Toks formalus modelio būsenų tikrinimas tinkrina modelio logines algortimo savybės, ir tai leidžia patikti retas sistemos busenas ir panaudos scenarijus, 
  apie kurios nevisada pagalvoma rašant algortimus, atliekant statinę kodo analizę ar testuojant \cite{newcombe2015amazon}.
  Šis metodas leidžia aptinkti galimas klaidas prieš išmeginant sistemą realioje aplinkose,
  taip sumažinant sistemos taisymo kaštus ir užkertant kritiniams sistemos sutrikimams,
  kas yra ypatingai svarbu kritinėse sistemose ar sistemose kurios siekia turėti dideli pasiekiamumą.

  Moksliniame straipsnyje ,,Effectiveness of the asynchronous client-side coordination of cluster service sessions'' aprašomas algortimas,
  galintis sumažinti į neveikianti mazga nukreiptų užklausų skaičių, stebinti sistemos klientų busenas 
  ir informuojant kitus sistomos klientus apie kliento aptinktus sutrikimus.
  Modelyje nenumatomi laikini tiklo sutrikimai, dėl kurių kai kurie mazgai siunčiantiems užklausas gali laikinai tapti klientams gali atrodyti nepasiekiami.
  Šis darbas nagrinėja algoritmo elgseną, papildydamas algortimo modelį laikinų sutrikimų būsenomis.


\subsubsection*{Darbo tikslas}

  Tikslas: Išanalizuoti sesijų valdymo algoritmą, įvertinti laikinių tinklo sutrikimų poveikį ir apžvelgti būdus kaip galima būtų patobulinti algoritmą.

\subsubsection*{Darbo uždaviniai}
    
  \begin{itemize}
    \item Atlikti mokslinės literatūros analizę, nagrinėjant panašius sesijų valdymo algoritmus.
    \item Išanalizuoti sesijų valdymo algoritmo modelį.
    \item Suformuluoto savybę, įrodancią, kad algoritmas veiks tinkamai, nepavyks prisijungti prie vienintelio aktyvaus tiekėjo mazgo.
    \item Naudojant TLA+ toolbox programinį paketą įrodyti, kad egzistuoja sistemos būsena kurioje pateikta savybė nėra tenkima.
    \item Pasiūlyti kaip būtu galima patobulininti algoritmą, kad pateikta savybė būtų tenkinama.
  \end{itemize}



\subsubsection*{Tyrimo metodai}

  Darbe atliekama asynchoninio sesijų valdymo algoritmo, aprašyto TLA\textsuperscript{+} modeliavimo kalba analizė, 
    praktiškai tikrinamas jo savybių tesingumas naudojant ,,TLA\textsuperscript{+} toolbox'' įrankį, 
    testuojant skirtingus modelio parametrus.
  Taip pat, darbe apžvelgiama paskistytų asynchroniškai bendraujančių systemų valdymo algortimų mokslinės literatūra. 

\subsubsection*{Darbo struktūra}

Pirmajame skyriuje aprašomas sesijų valdymo algoritmo modelis. 
Antrajame skyriuje aprašomas aprašyto modelio papildymas tinklo sutrikimais, ir poveikis sistemos modeliui. 
Trečiajame skyriuje nagrinėjama algoritmo patobulimo galimybės.